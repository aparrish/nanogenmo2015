\documentclass[12pt,openany]{book}
\frenchspacing
\usepackage{titlesec}
\titleformat{\chapter}[display]
  {\normalfont\huge\mdseries}{}{16px}{\Large}
\titlespacing*{\chapter}
  {0pt}{0pt}{16pt}

\begin{document}

\title{\textit{Our Arrival}: A Novel}
\author{Allison Parrish}
\date{November 2015}

\frontmatter
\maketitle

\pagestyle{empty}
%% copyrightpage
\begingroup
\footnotesize
\parindent 0pt
\parskip \baselineskip
\textcopyright{} 2015 Allison Parrish \\
All rights reserved.

This work is licensed with Creative Commons Attribution-ShareAlike 4.0
International. You are free to share (copy and redistribute the material in any
medium or format) and adapt (remix, transform, and build upon the material) for
any purpose, even commercially.

The licensor cannot revoke these freedoms as long as you follow the license
terms.

http://creativecommons.org/licenses/by-sa/4.0/

\endgroup
\clearpage

\pagestyle{myheadings}

\chapter{Preface}

This novel is a procedurally generated diary of an expedition through
fantastical places that do not exist.

The novel's primary source text is a database of over 5700 sentences drawn from
the Project Gutenberg corpus. Each sentence was selected based on semantic and
syntactic criteria, namely: the sentence must not have any nouns that refer to
human beings; the sentence must have as its subject some kind of natural object
or phenomena; the sentence must not have a pronoun as its subject; and the
sentence must be in the past tense. The resulting list of sentences are all
(more or less) assertions about the natural world. (The sentences are sourced
from a subset of Project Gutenberg books, namely those whose subject entries
include the strings \textit{Western}, \textit{Science fiction},
\textit{Geology}, \textit{Natural}, \textit{Exploration}, \textit{Discovery} or
\textit{Physical}.)

I created a number of different procedures to produce the sentences that
comprise the text of the novel. The most common of these works like this: two
sentences are selected at random from the sentence database (described above),
and parsed into their grammatical constituents. The \textit{subject noun
phrase} of the first sentence is replaced by the subject noun phrase of the
second sentence, resulting in a hybrid sentence combining aspects of both of
its sources. For example, these two sentences:

\begin{quote}
On either side \textit{the walls} were steep and rocky.\\
\textit{These sounds of woe} were full of meaning.
\end{quote}

... might be combined to form the following sentences:

\begin{quote}
On either side \textit{these sounds of woe} were steep and rocky.\\
\textit{The walls} were full of meaning.
\end{quote}

The text generation procedure remembers the \textit{topic} of each sentence,
and occasionally generates sentences that use a pronoun to refer back to and
elaborate on the subject of a previous sentence. A number of other textual
generation procedures produce expressions of emotion, awareness and affect on
the part of the novel's two main characters (\textit{I} and \textit{you}).

\mainmatter

<<chapters>>

\backmatter

\chapter{Attributions}

WordNet Release 3.0 This software and database is being provided to you, the
LICENSEE, by Princeton University under the following license. By obtaining,
using and/or copying this software and database, you agree that you have read,
understood, and will comply with these terms and conditions.: Permission to
use, copy, modify and distribute this software and database and its
documentation for any purpose and without fee or royalty is hereby granted,
provided that you agree to comply with the following copyright notice and
statements, including the disclaimer, and that the same appear on ALL copies of
the software, database and documentation, including modifications that you make
for internal use or for distribution. WordNet 3.0 Copyright 2006 by Princeton
University. All rights reserved. THIS SOFTWARE AND DATABASE IS PROVIDED "AS IS"
AND PRINCETON UNIVERSITY MAKES NO REPRESENTATIONS OR WARRANTIES, EXPRESS OR
IMPLIED. BY WAY OF EXAMPLE, BUT NOT LIMITATION, PRINCETON UNIVERSITY MAKES NO
REPRESENTATIONS OR WARRANTIES OF MERCHANT- ABILITY OR FITNESS FOR ANY
PARTICULAR PURPOSE OR THAT THE USE OF THE LICENSED SOFTWARE, DATABASE OR
DOCUMENTATION WILL NOT INFRINGE ANY THIRD PARTY PATENTS, COPYRIGHTS, TRADEMARKS
OR OTHER RIGHTS. The name of Princeton University or Princeton may not be used
in advertising or publicity pertaining to distribution of the software and/or
database. Title to copyright in this software, database and any associated
documentation shall at all times remain with Princeton University and LICENSEE
agrees to preserve same.

\end{document}
